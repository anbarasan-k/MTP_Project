% Introduction

% Main chapter title
\chapter{INTRODUCTION}
\label{chap:intro}

\section{Web Applications}
Web Applications are application programs that are stored in remote servers and delivered over the Internet using browser interfaces. The user interface of web applications are built using the interplay of HTML, CSS and JavaScript. Javascript being a highly dyanamic language, web applications are dynamic with interactive elements and some applications require server side processing. Web Applications require a server to handle client requests and a database to store the information. Web Applications include e-commerce websites, online forms, content management systems and email programs[1]. 

\section{Security Testing}
With the increasing complexity of web systems, and more content moving from offline to online platform, security testing has become a critical activity of web application development life cycle. The focus of security testing of web applications is to maintain the confidentiality of data, prevention against information leakages. This is performed by checking web application behaviour on injecting malicious data. Security Testings aids in exposing security vulnerabilities like XSS, SQL Injection, file inclusion, URL injection[3]. 

\section{Web Application Modelling}
A model is an abstraction or simplification of the behavior of the application under test(SUT). The model is captured in a machine with the purpose of acting as a test sequence (trace) generator. The model represents the architecture of the web application. A web application can be modelled using finite State machine or labelled transition system approaches[7]. 

\section{Model Based Testing}
Model-based testing is a technique for designing and executing applications to perform testing (includes Test cases to be executed on every object in the model). In Model Based Testing, test cases are generated automatically and systematically from the model of the system under test. There are three step in Model based Testing. Firstly, a model of the SUT is built from informal requirements or from the functionality of the SUT. Secondly, execution traces of the model are used to generate test cases. Since there can be infinitely many/long execution traces be present, test selection criteria is applied reduce the testcases. Lastly, using the test model and the test selection criteria, Test cases are derived for model based testing[7]. 

\section{Model Based Security Testing}

Model based testing can be adapted to test security of web applications. Model based testing for functionality usually generates positive test cases for validation of web applications. To perform security testing, negative test cases and attack scenarios have to be derived. Attack scenarios include generation of malicious code or data to retrieve unauthorized data from the web server. This involves two steps- Modeling activity to capture the behavioral aspects and/or architecture of the web application. The next step is to define the test purposes- in this case to test the security in particular the type of attack on web security.   
 