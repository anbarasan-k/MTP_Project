
\chapter{PROBLEM STATEMENT}

The growth of Internet and evolution of web application technologies has led to the vulnerabilities in security- data confidentiality, data integrity and service availability. Manual testing of security in web applications is time consuming and its difficult to perform in case of regression testing. Web Vulnerability scanners can be used to detect vulnerabilities but they often led to generation of false positive and false negative results. A desirable solution to perform security testing is automated security testing. 
Automated security testing has its challenges in terms of deriving proper functional behavior of the web applications. This is due to the lack of proper documentation of web applications or various technologies used for developing web applications. Modern web applications are no more static but dynamic due to the extensive use of javascript and CSS. With javascript, functionalities can be added in the client side of the web applications which makes calls to the server using (Asynchronous javascript) AJAX. These calls needs to be validated and tested thoroughly to prevent information leaks. 

\newline
To overcome the above difficulties in security testing of application, there is a need of method for automated testing of vulnerabilities of software that can be realized testing it with a model can overcome the above challenges. This approach is called model based security testing. We derive a model for web applications handling AJAX(Javascript and XML) calls and test security vulnerabilities based on the testcases derived from model.

\newline
The thesis is organized as follows:\\
Chapter 3- Literature Survey, Chapter 4- Web Application Security Vulnerabilities, Chapter5- Our approach of modelling web applications for security testing, Chapter 6- Modelling of Web Applications via State Machine Chapter 7- Security Test Framework, Chapter 8- Experimentation and Results, Chapter 9- Related Work, Chapter 10- Conclusion and Chapter 11- Future Work

Existing approaches like SQLMap performs security testing without out a model. Our approach being model driven provides exhaustive coverage of testing web applications.
